\documentclass[11pt,a4paper]{article}

\usepackage[utf8]{inputenc}
\usepackage[T1]{fontenc}
\usepackage{lmodern}
\usepackage{graphicx}
\usepackage{xcolor}
\usepackage{hyperref}
\usepackage{amsmath,amssymb}
\usepackage{geometry}

\geometry{
  left=2.5cm,
  right=2.5cm,
  top=3cm,
  bottom=3cm
}

\title{\textbf{ORSD for Deliberation Knowledge Graph: Papers, Ontologies, and Frameworks}}
\author{Simone Vagnoni\thanks{CIRSFID, University of Bologna; OEG, Universidad Politecnica de Madrid}}
\date{\today}

\begin{document}
\maketitle


\tableofcontents

\section{Introduction}
This report addresses a collection of references relevant to e-Participation, deliberation, legal frameworks, and argumentation modeling. Certain references point to actual \emph{ontologies} expressed in RDF/OWL, while others are \emph{standards, formats}, or \emph{frameworks} that are not strictly ontological but are often essential to the modeling of deliberative or legal processes. 

In Section~\ref{sec:whichOntologies}, we briefly classify which references represent actual ontologies and which do not. Section~\ref{sec:orsdRefined} then proposes a refined ORSD with separate subsections for (i) ontologies, (ii) standard formats, and (iii) frameworks or conceptual methods. Pros and cons of each are summarized to inform the usage in future projects.


\section{ORSD Draft}\label{sec:orsdRefined}

Below is a proposed reorganization of the ORSD (Ontology Requirements Specification Document) for the Deliberation Knowledge Graph. We separate actual \emph{ontologies} from other \emph{standards} and \emph{frameworks} to clarify their different nature in the domain.

\subsection{Ontologies}

\subsubsection{DELIB Ontology \cite{PanagiotopoulosGionis2011}}
\textbf{Purpose:} Captures core elements of deliberative processes, participant interactions, and outcomes.\\
\textbf{Pros:} Offers explicit classes for contributions, arguments, and deliberation stages; fosters integration of diverse data.\\
\textbf{Cons:} Less detail on advanced legal aspects; fewer real-world adoption scenarios are documented.

\subsubsection{Deliberation Ontology (Panagiotopoulos et al.) \cite{PanagiotopoulosSamsElliman2011}}
\textbf{Purpose:} Focused on supporting public decision-making in policy contexts, bridging arguments and legal references.\\
\textbf{Pros:} Integrates legal info strongly and models bureaucratic elements clearly.\\
\textbf{Cons:} Mostly government-centric, with limited modeling of informal spaces and grassroots deliberation.

\subsubsection{SIOC (Semantically Interlinked Online Communities) \cite{SIOC2006}}
\textbf{Purpose:} Well-known ontology for describing online discussions, user roles, threads, etc.\\
\textbf{Pros:} Standard widely used in social media and community platforms, straightforward to adopt.\\
\textbf{Cons:} Not originally designed for a “political/legal” angle; must be extended for advanced deliberation or argumentation uses.

\subsubsection{AIF (Argument Interchange Format) \cite{Chesnevar2006}}
\textbf{Purpose:} RDF-based structure for modeling arguments (premises, conclusions), plus relationships like support and attack.\\
\textbf{Pros:} Deep theoretical foundation in argumentation; facilitates tool interoperability.\\
\textbf{Cons:} Arg-focused only; does not cover participant roles, timelines, or broader deliberation processes.

\subsubsection{LKIF (Legal Knowledge Interchange Format) \cite{Hoekstra2007}}
\textbf{Purpose:} Models legal norms, rules, concepts for advanced legal knowledge representation.\\
\textbf{Pros:} Comprehensive coverage of legal reasoning and representation.\\
\textbf{Cons:} Overkill for non-legal contexts; can be quite specialized and complex to deploy.

\subsubsection{IBIS (Issue-Based Information System) \cite{Kunz1970}}
\textbf{Purpose:} Simple approach to capturing issues, positions, and arguments (design rationale approach).\\
\textbf{Pros:} Clear, intuitive structure widely recognized in decision support contexts.\\
\textbf{Cons:} Less expressive for large-scale or multi-actor deliberation; lacks participant or institutional modeling.

\subsubsection{Ontology for Next Generation e-Participation (Porwol et al.) \cite{Porwol2016}}
\textbf{Purpose:} Integrates e-Participation domain aspects (process, project, platform) with special attention to citizen-led + government-led synergy.\\
\textbf{Pros:} Broad conceptual coverage, includes dynamic social media interplay.\\
\textbf{Cons:} Lacks robust examples of large multi-platform usage or deeply integrated legal frameworks.

\subsubsection{Social Participation Ontology (Fabbri et al.) \cite{Fabbri2015}}
\textbf{Purpose:} Defines concepts for bridging formal and informal participation, linking multi-level civic engagement data.\\
\textbf{Pros:} Facilitates synergy between top-down and grassroots contexts; interoperable with FOAF, etc.\\
\textbf{Cons:} Very broad scope can complicate adoption for narrower or specialized deliberation tasks.

\subsubsection{Core Ontology on Decision Making (Guizzardi et al.) \cite{GuizzardiDecision2020}}
\textbf{Purpose:} A UFO-based conceptualization of decisions, intentions, preferences, and values.\\
\textbf{Pros:} Highly rigorous; clarifies the micro-level intentions and commitments underlying decisions.\\
\textbf{Cons:} Additional layers needed for socio-political processes, large-scale e-Participation data integration.


\subsection{Standards or Formats (Non-Ontological)}

\subsubsection{Akoma Ntoso \cite{AkomaNtoso}}
\textbf{Form:} XML/LegalDocML standard.\\
\textbf{Purpose:} Interoperable representations of parliamentary, legislative, and judiciary documents.\\
\textbf{Pros:} Widely recognized for legislative publishing, strong in structuring legal texts.\\
\textbf{Cons:} Not an RDF/OWL ontology, lacking direct semantic relations for advanced knowledge modeling.

\subsubsection{Metalex \cite{Metalex}}
\textbf{Form:} XML-based standard.\\
\textbf{Purpose:} Aims to ensure consistent legislative text interchange, referencing norms.\\
\textbf{Pros:} Facilitates cross-jurisdiction standardization of normative documents.\\
\textbf{Cons:} Focused on textual structure rather than conceptual modeling for arguments or deliberation.

\subsubsection{LegalRuleML \cite{Athan2015}}
\textbf{Form:} Formalism for expressing legal rules and reasoning structures.\\
\textbf{Purpose:} Captures normative logic, conditions, and advanced legal arguments.\\
\textbf{Pros:} Powerful for complex rule-based legal reasoning, machine-processable.\\
\textbf{Cons:} Not an OWL ontology per se; advanced knowledge needed for integration with e-Participation ontologies.

\subsection{Frameworks or Conceptual Models}

\subsubsection{Ontological Models of Legal Contents and Users’ Activities (Schmitz et al.) \cite{Schmitz2017}}
\textbf{Type:} Model-based approach with modular knowledge organization for EU e-Participation.\\
\textbf{Pros:} Tightly integrates legal documents (FRBR, EuroVoc) with user feedback in multiple languages.\\
\textbf{Cons:} Early pilot stage, heavily EU-centric.

\subsubsection{PAKT (Perspectivized Argumentation Knowledge Graph) (Plenz et al.) \cite{Plenz2024}}
\textbf{Type:} Knowledge Graph + analytics tool, not purely an ontology.\\
\textbf{Pros:} Multi-layered approach capturing argument structure, frames, underlying values.\\
\textbf{Cons:} Implementation can be complex, reliant on robust NLP pipelines.

\subsubsection{Modelling e-Participation Implementation: A Network-Based Approach (Rossello et al.) \cite{Rossello2023}}
\textbf{Type:} Method/Analytical approach using Social Network Analysis.\\
\textbf{Pros:} Clarifies hybrid offline-online workflows and stakeholder interactions.\\
\textbf{Cons:} Lacks direct semantic standardization; bridging with e-Participation ontologies would be manual.




\section{Why a New Deliberation Knowledge Graph is Needed}

Despite the availability of multiple ontologies in related domains, a unified Deliberation Knowledge Graph is necessary for several reasons:

\begin{enumerate}
    \item \textbf{Integration Gap:} None of the existing ontologies adequately bridges formal institutional deliberation with civic participation platforms. The DKG addresses this by creating mappings between these domains.
    
    \item \textbf{Fallacy Detection Support:} Existing ontologies lack the specific structures required for computational identification of logical fallacies in deliberative discourse. The DKG explicitly models argument patterns needed for automated fallacy detection.
    
    \item \textbf{Cross-Dataset Standardization:} Current deliberation data exists in heterogeneous formats across platforms. The DKG provides a common semantic framework to normalize and integrate these diverse datasets.
    
    \item \textbf{Fragmentation:} Each existing ontology covers only part of the deliberation ecosystem. The DKG provides a comprehensive framework that connects process, participants, content, and information in a coherent structure.
    
    \item \textbf{Interoperability Challenges:} Current solutions operate in silos, making cross-platform analysis difficult. The DKG establishes common semantics to enable data sharing across different deliberation environments.
    
    \item \textbf{Multi-perspective Integration:} Existing approaches typically adopt either a government-centric or citizen-centric perspective. The DKG supports multiple viewpoints simultaneously.
    
    \item \textbf{Technical Evolution:} New deliberation platforms and technologies emerge regularly. The DKG's modular approach allows extension to incorporate new deliberation forms while maintaining backward compatibility.
    
    \item \textbf{Research-Practice Gap:} Current ontologies are either too theoretical or too implementation-specific. The DKG balances conceptual rigor with practical applicability.
\end{enumerate}

The Deliberation Knowledge Graph addresses these limitations by providing a unifying semantic layer that leverages the strengths of existing ontologies while filling their gaps through a comprehensive approach to deliberation modeling. Specifically, it adds crucial support for fallacy detection and cross-dataset integration that has been missing from previous ontological approaches.

\bibliographystyle{IEEEtran}
\bibliography{references}

% FIRST TABLE
\begin{table}
\centering
\scriptsize
\begin{tabular}{| l | l | l | l  | l | l | l |l| }
\hline
\multicolumn{8}{|c|}{\textbf{Deliberation Knowledge Graph Ontology Requirements Specification Document (Part 1)}}                                                                                                                                                                                                                                                         \\ \hline
\multicolumn{8}{|c|}{\textbf{1. Purpose}}                                                                                                                                                                                                                                                         \\ \hline
\multicolumn{8}{| p{14.0cm} |}{The purpose of the Deliberation Knowledge Graph is to create a comprehensive, interoperable data moDKG that represents deliberative processes across different platforms and contexts. This knowledge graph aims to structure and connect the various elements of deliberations (participants, arguments, processes, and information) in a way that enables analysis, visualization, and comparison of deliberative activities across platforms, from formal parliamentary debates to citizen participation initiatives. The ontology specifically addresses requirements for logical fallacy detection in deliberative discourse and provides a unified representation of common elements across diverse deliberation datasets.}                                                                                                                                                                                                                                                      \\ \hline
\multicolumn{8}{|c|}{\textbf{2. Scope}}                                                                                                                                                                                                                                                         \\ \hline
\multicolumn{8}{| p{14.0cm} |}{The ontology covers the domain of deliberative processes in both institutional and civic contexts. It encompasses:

\begin{itemize}
\item Formal deliberation processes (e.g., parliamentary debates, legislative procedures)
\item Participatory deliberation platforms (e.g., Decidim, civic consultation tools)
\item Arguments and contributions exchanged during deliberations
\item Participants and their roles in deliberative processes
\item Information resources that inform or result from deliberations
\item Temporal and thematic organization of deliberative content
\item Logical fallacy patterns and classifications
\item Common structures and fields across different deliberation platforms and datasets
\item Standard argument components required for fallacy detection
\end{itemize}

The ontology explicitly excludes:
\begin{itemize}
\item Detailed representation of document content beyond what's relevant to deliberation
\item Internal platform-specific technical details not related to deliberative processes
\item Implementation details of fallacy detection algorithms (focusing instead on required data structures)
\end{itemize}}                                                                                                                                                                                                                                                      \\ \hline
\multicolumn{8}{|c|}{\textbf{3. Implementation Language}}                                                                                                                                                                                                                                                         \\ \hline
\multicolumn{8}{| p{14.0cm} |}{The ontology will be implemented in OWL 2 (Web Ontology Language) using RDF/XML syntax to ensure maximum compatibility with existing semantic web technologies. The ontology will follow W3C standards and best practices for linked data. SPARQL will be used for query specifications for fallacy detection patterns.}                                                                                                                                                                                                                                                      \\ \hline
\multicolumn{8}{|c|}{\textbf{4. Intended End-Users}}                                                                                                                                                                                                                                                         \\ \hline
\multicolumn{8}{| p{14.0cm} |}{User 1. Government officials and policy makers who need to analyze and understand deliberative processes.\newline
User 2. Civic participation platform administrators who want to integrate deliberative content across platforms.\newline
User 3. Researchers studying deliberative democracy and analyzing deliberative data.\newline
User 4. Developers creating applications that connect or visualize deliberative content from multiple sources.\newline
User 5. Citizens and civic organizations seeking to understand and participate in deliberative processes.\newline
User 6. Fallacy detection system developers who require structured argument data.\newline
User 7. Cross-platform data analysts who need to work with multiple deliberation datasets.}                                                                                                                                                                                                                                                      \\ \hline
\multicolumn{8}{|c|}{\textbf{5. Intended Uses}}                                                                                                                                                                                                                                                         \\ \hline
\multicolumn{8}{| p{14.0cm} |}{Use 1. Integration of deliberative content from multiple platforms and sources to create unified views of participation.\newline
Use 2. Analysis of deliberative processes to understand participation patterns, argument flows, and information usage.\newline
Use 3. Visualization of deliberation structures and their evolution over time.\newline
Use 4. Importing/exporting deliberative content between different systems (e.g., from Decidim to parliamentary systems).\newline
Use 5. Creation of searchable repositories of deliberative content with rich semantic relationships.\newline
Use 6. Enabling the development of tools to support more effective deliberation processes.\newline
Use 7. Automated detection and flagging of logical fallacies in arguments.\newline
Use 8. Standardized representation of deliberation data from heterogeneous sources.\newline
Use 9. Training and validation of AI models for argument analysis.}                                                                                                                                                                                                                                                      \\ \hline
\multicolumn{8}{|c|}{\textbf{6. Ontology Requirements}}                                                                                                                                                                                                                                                         \\ \hline
\multicolumn{8}{|c|}{\textbf{a. Non-Functional Requirements}}                                                                                                                                                                                                                                                         \\ \hline
\multicolumn{8}{| p{14.0cm} |}{NFR 1. The ontology should be compatible with existing deliberation platforms' data models (Decidim, parliamentary data, etc.).\newline
NFR 2. The ontology should be extensible to accommodate new deliberation platforms and formats.\newline
NFR 3. The ontology should reuse existing ontologies where appropriate (SIOC, FOAF, SKOS, etc.).\newline
NFR 4. The ontology should be documented with clear examples of usage.\newline
NFR 5. The ontology should support multilingual content.\newline
NFR 6. The ontology should provide mappings to previous deliberation ontologies (DELIB, AIF, etc.).\newline
NFR 7. The ontology should support reasoning for automated fallacy detection.\newline
NFR 8. The ontology should maintain consistency across data from different sources.}                                                                                                                                                                                                                                                      \\ \hline
\end{tabular}
\caption{Ontology Requirements Specification Document for Deliberation Knowledge Graph (Part 1)}
\label{tab:ORSD-1}
\vspace{-0.1in}
\end{table}

\begin{table}[ht]
\centering
\scriptsize
\begin{tabular}{|l|l|l|l|l|l|l|l|}
\hline
\multicolumn{8}{|c|}{\textbf{Deliberation Knowledge Graph Ontology Requirements Specification Document (Part 2)}} \\
\hline
\multicolumn{8}{|c|}{\textbf{6. Ontology Requirements (continued)}} \\
\hline
\multicolumn{8}{|c|}{\textbf{b. Functional Requirements: Groups of Competency Questions}} \\
\hline
\multicolumn{4}{|c|}{CQG1. Deliberation Process Structure} & \multicolumn{4}{c|}{CQG2. Participant Information} \\
\hline
\multicolumn{4}{|p{6.8cm}|}{
CQ1. What are the stages of a specific deliberation process?\newline
CQ2. When did a particular deliberation start and end?\newline
CQ3. What deliberation processes exist on a particular topic?\newline
CQ4. What are the participation rules for a deliberation process?\newline
CQ5. Which organization or entity is responsible for a specific deliberation process?
} &
\multicolumn{4}{p{6.8cm}|}{
CQ6. Who are the participants in a specific deliberation?\newline
CQ7. What is the role of a participant in a deliberation process?\newline
CQ8. In which deliberations has a particular participant contributed?\newline
CQ9. Which organizations are represented in a deliberation?\newline
CQ10. How many participants were involved in a deliberation process?
} \\
\hline
\multicolumn{4}{|c|}{CQG3. Contributions and Arguments} & \multicolumn{4}{c|}{CQG4. Information Resources} \\
\hline
\multicolumn{4}{|p{6.8cm}|}{
CQ11. What contributions were made in a specific deliberation?\newline
CQ12. Who made a specific contribution or argument?\newline
CQ13. Which arguments support or oppose a specific position?\newline
CQ14. What is the thread structure of a deliberative conversation?\newline
CQ15. Which contributions received the most responses or engagement?\newline
CQ16. What information sources are referenced in contributions?
} &
\multicolumn{4}{p{6.8cm}|}{
CQ17. What legal documents are referenced in a deliberation?\newline
CQ18. What information frameworks are relevant to a specific deliberation?\newline
CQ19. What documents were produced as a result of a deliberation?\newline
CQ20. How is legal information structured and referenced in deliberations?
} \\
\hline
\multicolumn{4}{|c|}{CQG5. Fallacy Detection} & \multicolumn{4}{c|}{CQG6. Cross-Dataset Integration} \\
\hline
\multicolumn{4}{|p{6.8cm}|}{
CQ21. What logical fallacies can be identified in a specific argument?\newline
CQ22. What are the premise-conclusion relationships in an argument?\newline
CQ23. Which arguments contain ad hominem attacks?\newline
CQ24. What patterns of circular reasoning exist in a deliberation?\newline
CQ25. How can arguments be classified by fallacy type?\newline
CQ26. What evidence is provided to support a claim?
} &
\multicolumn{4}{p{6.8cm}|}{
CQ27. What common fields exist across different deliberation platforms?\newline
CQ28. How can participant identities be reconciled across platforms?\newline
CQ29. What standard argument elements can be mapped across datasets?\newline
CQ30. How are temporal aspects of deliberation represented consistently?\newline
CQ31. What minimal data structure is required for cross-platform analysis?
} \\
\hline
\multicolumn{8}{|c|}{\textbf{7. Pre-Glossary of Terms}} \\
\hline
\multicolumn{8}{|c|}{\textbf{a. Terms from Competency Questions + Frequency}} \\
\hline
\multicolumn{1}{|p{3.15cm}|}{
Deliberation (15)\newline
Argument (9)\newline
Process (8)
} &
\multicolumn{1}{p{3.15cm}|}{
Participant (7)\newline
Contribution (6)\newline
Information (5)
} &
\multicolumn{1}{p{3.15cm}|}{
Fallacy (5)\newline
Platform (4)\newline
Dataset (4)
} &
\multicolumn{5}{p{3.15cm}|}{
Legal (4)\newline
Document (3)\newline
Structure (3)
} \\
\hline
\multicolumn{8}{|c|}{\textbf{b. Objects}} \\
\hline
\multicolumn{8}{|p{14cm}|}{
DeliberationProcess, Stage, Activity, Timeline, Decision, Implementation, Participant, Role, Organization, Group, Contribution, Argument, Position, Vote, Opinion, Thread, Topic, InformationResource, LegalSource, LegalDecomposition, LegalFramework, LegalInterpretation, FallacyType, ArgumentStructure, Premise, Conclusion, Evidence, CrossPlatformIdentifier, CommonField, StandardStructure
} \\
\hline
\end{tabular}
\caption{Ontology Requirements Specification Document for Deliberation Knowledge Graph (Part 2)}
\label{tab:ORSD-2}
\end{table}



\end{document}